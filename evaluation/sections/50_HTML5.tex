\section{HTML5}
Die HTML5 Version der App ist großteils identisch mit der Phonegap Version. Die Entwicklung der beiden Versionen erfolgte in enger Abstimmung, um einen Vergleich zwischen einer reinen HTML5 App und der Verwendung von Phonegap zur Umwandlung in native Apps unternehmen zu können. Zu Beginn der Entwicklung verwendeten wir zunächst Bootstrap als Framework für die beiden Versionen. Dies ermöglichte es die Kern-Features in den ersten Wochen zügig zu implementieren. Nach gut einem Monat Entwicklung erwies sich Bootstrap alleine jedoch als ungeeignet, da es die App nicht von sich aus strukturierte und zudem die Phonegap Version zu diesem Zeitpunkt nicht mit Windows Phone kompatibel war. Daher erfolgte in Absprache mit den Projekt-Initiatoren ein Umstieg auf AngularJS, spezifischer Mobile Angular UI. Trotz des hohen  Aufwandes, die App über zwei Wochen auf ein anderes Framework umzuziehen und dabei parallel weiter zu entwickeln, zahlte sich die Entscheidung aus. Die Phonegap Version wurde dadurch kompatibel zu Windows und die weitere Entwicklung der beiden Versionen beschleunigt. Zur Verwaltung der Plugins nutzten wir Bower, sowie Gulp als Build Tool. Als Testframework kam Karma in Kombination mit Jasmine zum Einsatz.

\subsection{IDE}
Aufgrund der großen Auswahl vergeben wir für die IDE bei der HTML5 Version der App die volle Punktzahl. Wir verwendeten für die Erstellung der HTML5 App Sublime Text 3. Allerdings kann je nach persönlicher Präferenz ein beliebiger Editor verwendet werden. Dadurch ergibt sich bei der Entwicklung keine Einschränkung bezüglich des Betriebssystems und es kann auf kostenfreie Software zurückgegriffen werden. Dies erleichtert des Weiteren den Einstieg sowie die Einrichtung eines Entwickler-Arbeitsplatzes.

\subsection{Language}
Für HTML5 als Sprache vergeben wir 3 von 3 Punkten. Die Sprache profitiert von ihrer langen Entstehungsgeschichte und zeichnet sich zugleich durch ihren Umfang, als auch ihre Zugänglichkeit aus. Zudem ist sie komplatibel mit vielen anderen Sprachen aus dem Kontext der Web-Entwicklung, was sich positiv auf die Erweiterbarkeit einer HTML5 App auswirkt. Im Rahmen dieses Projektes erweiterten wir die HTML5 App mit Javascript. Dies war durch die Vorgabe kein Backend zu verwenden notwendig geworden, um einen großen Teil der eigentlichen Funktionalität der App zu realisieren. Die Verwendung von CSS3 zur einheitlichen grafischen Gestaltung der App führen wir hier aufgrund der engen Verwandtschaft zu HTML5 nicht weiter aus.

\subsection{Support}
Für den Support vergeben wir alle 3 Punkte. Als vom W3C vorangetriebene Sprache profitiert HTML5 von einer umfangreichen Dokumentation, sowie zahlreichen Tutorials, Blogs und Foren, die sich mit der Sprache beschäftigen. Allerdings ist es dabei wichtig zu beachten, dass HTML5 in den allermeisten Fällen zur Web-Entwicklung eingesetzt wird. Die Verwendung als App auf einem Smartphone oder Tablet stellt derzeit noch eine Niche dar, was sich auch an der Anzahl und Qualität der online zu findenden Hilfe bemerkbar macht.

\subsection{Geolocation}
Für die Möglichkeiten zur Geolocation vergeben wir die volle Punktzahl. HTML5 beinhaltet eine entsprechende API, welche von allen aktuellen Browsern unterstützt wird. Jedoch erlauben die Browser die Verwendung dieser Möglichkeit nur nach expliziter Erlaubnis durch den Nutzer. Dies ist Enwickler-seitig nicht zu umgehen. Die Browser verwenden zur Geolocation je nach Verfügbarkeit und Implementierung durch den Browser-Entwickler Informationen aus der IP-Adresse, dem WLAN, Funkzelleninformationen der Mobilfunknetze und GPS. Wenn dies gewünscht ist, kann der HTML5-Entwickler allerdings die alleinige Verwendung der GPS-Daten vorschreiben.

\subsection{Notifications}
Für die Verwendung lokaler Benachrichtigungen auf mobilen Endgeräten vergeben wir keine Punkte. HTML5 allein bietet für derartige Benachrichtigungen keine Möglichkeit. Um dies dennoch mit HTML5 umsetzen zu können, muss man auf zusätzliche Software zurückgreifen, die aus der HTML5-App native Apps für die jeweiligen Betriebssysteme generiert. Ein Beispiel dafür stellt Phonegap dar, welches im Rahmen dieses Projektes ebenso evaluiert wurde.

\subsection{Email}
Für die Möglichkeit zur Versendung von Emails vergeben wir 1 von 3 Punkten. Der in diesem Projekt vorgesehene Anwendungsfall zur Evaluierung dieser Möglichkeit konnte mit HTML5 nicht umgesetzt werden. Zwar bietet HTML5 die Möglichkeit das jeweilige Standard-Email-Programm zu öffnen und diesem einige Parameter zu übergeben, das in Anwendungsfall enthaltene Beifügen eines Anhangs ist aber standardmäßig nicht möglich. Die Verwendung eines Backends, in diesem konkreten Fall einem Mail-Server, hätte die Umsetzung des Features erlaubt, da HTML5 die Verwendung von Websockets unterstützt.

\subsection{Persistence}
Für die Möglichkeiten zur persistenten Speicherung vergeben wir 2 von 3 Punkten. HTML5 bietet mehrere Möglichkeiten zur Speicherung von Daten, die nicht alle persistenter Natur sind. Zur persistenten Speicherung wird SQLite verwendet. Das HTML5 eigene Web SQL wird vom W3C als veraltet bezeichnet und nicht weiter entwickelt.

\subsection{Design}
Für die Möglichkeiten zur Gestaltung des Designs vergeben wir alle 3 Punkte. HTML5 in Kombination mit CSS3 und ggf. auch Javascript ermöglichen in einer App all die aus der Web-Entwicklung bekannten Gestaltungsmöglichkeiten. Zudem existieren hierzu schon zahlreiche Frameworks, womit auch komplexere Design Konzepte zügig umgesetzt werden können. Allerdings wird die Windows Phone Platform nur von wenigen Frameworks unterstützt, weshalb wir uns für das UI Framework Mobile Angular UI\footnote{\url{http://mobileangularui.com/}} entschieden, welches auf Angular JS und Bootstrap aufbaut.

\subsection{Testing}
Für die Möglichkeiten zum Testen einer HTML5 App vergeben wir 2 von möglichen 3 Punkten. Zwar gibt es hierzu verschiedene Testframeworks, jedoch sind diese zumeist auf das Testen von HTML5 im Rahmen der klassischen Web-Entwicklung spezialisiert, in dem ein Backend Teil des Systems ist. Frameworks speziell für das Testen von HTML5 Apps haben sich noch nicht verbreitet.
