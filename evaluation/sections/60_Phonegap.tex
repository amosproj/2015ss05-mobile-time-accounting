\section{Phonegap}
Die Phonegap Version der App ist zum Großteil identisch mit der HTML5 Version. Daher konzentriert sich die Evaluierung der Phonegap Version auf die Unterschiede zu dieser und verweist ansonsten auf die Evaluierung der HTML5 Version. Mit Hilfe des Phonegap/Cordova-Frameworks wird die App in hybride Versionen für Android, iOS und Windows Phone umgewandelt. Im Unterschied zur HTML5 Version kommen daher mehrere Phonegap-Plugins zum Einsatz, die Funktionalitäten ermöglichen, die sonst nur nativen Apps vorbehalten sind.

\subsection{IDE}
Wir vergeben für die IDE bei der Phonegap Version der App analog zur HTML5 Version die volle Punktzahl. Für das Builden der einzelnen Apps ist die Installation des Software Development Kits der jeweiligen Platform notwendig. Allerdings ist die Benutzung der mitgelieferten IDEs nur für das Deployen auf ein Endgerät oder dem Tracken von platformspezifischen Bugs erforderlich.

\subsection{Language}
Wir vergeben für die Sprache(n) bei der Phonegap Version der App analog zur HTML5 Version 3 von 3 Punkten.

\subsection{Support}
Wir vergeben für den Support bei der Phonegap Entwicklung 2 von 3 Punkten. Die von Apache zur Verfügung gestellte Dokumentation ist relativ umfangreich, jedoch ist sie nicht sehr übersichtlich und es mangelt an konkreten hilfreichen Beispielen. Darüber hinaus ist die Anzahl an Supportthreads im Internet sehr gering und nicht mit der Community von Android oder iOS vergleichbar. Bei webspezifischen und Phonegap/Cordova unabhängigen Problemen ist der Support, wie in der HTML Version beschrieben, sehr umfangreich.

\subsection{Geolocation}
Wir vergeben für die Geolocation bei der Phonegap Version analog zur HTML5 Version alle 3 Punkte. Das verwendete Cordova Geolocation Plugin wurde von Apache direkt entwickelt und basiert auf der W3C Geolocation API Specification\footnote{\url{http://dev.w3.org/geo/api/spec-source.html}}. Das Plugin ist sehr gut dokumentiert\footnote{\url{https://github.com/apache/cordova-plugin-geolocation}} und arbeitet prinzipiell ähnlich wie die HTML Variante.

\subsection{Notifications}
Wir vergeben für die Möglichkeit zur Nutzung lokaler Notifikationen bei der Phonegap Entwicklung 2 von 3 Punkten. Das verwendete Plugin für Notifikationen ist zwar gut dokumentiert\footnote{\url{https://github.com/katzer/cordova-plugin-local-notifications}}, allerdings ist es in der Anwendung auf die Grenzen der jeweiligen Platform für lokale Notifikationen beschränkt.

\subsection{Email}
Für die Möglichkeit zur Erzeugung von Emails auf mobilen Endgeräten vergeben wir die volle Punktzahl. Das verwendete Cordova Email Plugin\footnote{\url{https://github.com/katzer/cordova-plugin-email-composer}} ist ebenfalls gut dokumentiert und hat bei der Entwicklung keinerlei Probleme verursacht. Hervorzuheben ist das unkomplizierte Hinzufügen von Anhängen als base64 file stream oder vom lokalen Speicher.

\subsection{Persistence}
Für die Persistente Speicherung vergeben wir analog zur HTML5 Version 2 von 3 Punkten. Die Phonegap Version benutzt ein Cordova Sqlite Storage Plugin\footnote{\url{https://github.com/litehelpers/Cordova-sqlite-storage}}, welches sich lediglich beim Aufruf der openDatabase Funktion von der HTML Version unterscheidet. Ansonsten sind die Schnittstellen zur Datenbank identisch mit der HTML5/Web SQL API.

\subsection{Design}
Für das Design vergeben wir analog zur HTML5 Version alle 3 Punkte.

\subsection{Testing}
Für das Testen vergeben wir analog zur HTML5 Version 2 von 3 Punkten.
