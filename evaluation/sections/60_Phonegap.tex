\section{Phonegap}
Die Phonegap Version der App ist zum Großteil identisch mit der HTML5 Version. Daher konzentriert sich die Evaluierung der Phonegap Version auf die Unterschiede zu dieser und verweist ansonsten auf die Evaluierung der HTML5 Version. Mit Hilfe des Phonegap/Cordova-Frameworks wird die App in hybride Versionen für Android, iOS und Windows Phone umgewandelt. Im Unterschied zur HTML5 Version kommen daher mehrere Phonegap-Plugins zum Einsatz, die Funktionalitäten ermöglichen, die sonst nur nativen Apps vorbehalten sind.

\subsection{IDE}
Wir vergeben für die IDE bei der Phonegap Version der App analog zur HTML5 Version die volle Punktzahl.

\subsection{Language}
Wir vergeben für die Sprache(n) bei der Phonegap Version der App analog zur HTML5 Version 3 von 3 Punkten.

\subsection{Support}
Wir vergeben für den Support bei der Phonegap Entwicklung 2 von 3 Punkten. //TODO

\subsection{Geolocation}
Wir vergeben für die Geolocation bei der Phonegap Version analog zur HTML5 Version alle 3 Punkte. Neben der Nutzung der HTML5 Geolocation API ermöglicht Phonegap mit Plugins... //TODO

\subsection{Notifications}
Wir vergeben für die Möglichkeit zur Nutzung lokaler Notifikationen bei der Phonegap Entwicklung 2 von 3 Punkten. //TODO

\subsection{Email}
Für die Möglichkeit zur Erzeugung von Emails auf mobilen Endgeräten vergeben wir die volle Punktzahl. //TODO

\subsection{Persistence}
Für die Persistente Speicherung vergeben wir analog zur HTML5 Version 2 von 3 Punkten.

\subsection{Design}
Für das Design vergeben wir analog zur HTML5 Version alle 3 Punkte.

\subsection{Testing}
Für das Testen vergeben wir analog zur HTML5 Version 2 von 3 Punkten.
