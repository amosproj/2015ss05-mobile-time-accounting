\section{Windows}

Das native Betriebssystem Windows Phone besitzt einen verhältnismäßig kleinen Marktanteil gegenüber iOS und Android. Aufgrund des intensiven Engagements seitens Microsoft und der baldigen Veröffentlichung von Windows 10 (Desktop und Mobile), sollen dennoch Stärken und Schwächen analysiert werden. Für die Evaluierung wurde die neuste Version (Windows Phone 8.1) herangezogen.

\subsection{IDE}

Das Entwickeln von Windows Phone Apps ist ausschließlich mit Visual Studio 2013 (VS) möglich. Die notwendigen Komponenten werden beim sogenannten „Update 2“ mit installiert. Für viele Windows Entwickler ist die Extension „ReSharper“ von der Firma JetBrains essenziel um mit VS produktiv zu arbeiten. Externe Geräte werden sofort von der DIE erkannt und entsprechendes Debugging ist mittels VS sehr elegant gelöst. Im allgemeinen lässt sich mit diesen Entwicklungswerkzeugen sehr gut arbeiten, jedoch sind diese Tools gegenüber anderen Plattformen nicht kostenfrei (ReSharper: 249 Euro, VS Pro: 636 Euro).

\subsection{Language}

Bei Windows ist es üblich sich zwischen Visual Basic oder C\# als Programmiersprache zu entscheiden. Im Rahmen des Projekts wurde ausschließlich C\# verwendet. Solch eine mächtige objekt-orientierte Programmiersprache erleichtert den Entwicklungsaufwand, vorausgesetzt die Entwickler beherrschen diese. Zusätzlich werden Windows Desktop Apps ebenfalls mit C\# entwickelt, weshalb bei Projekten die beide Formate anbieten wollen oder wenn eine Desktop App schon entwickelt wurde, deutlich kürzere Entwicklungszeiten erwartet werden können.

\subsection{Support}

Obwohl Microsoft eine große Auswahl an Tutorials online anbietet und eine Vielzahl an Videos in der sogenannten Microsoft Virtual Academy zur Verfügung stellt wird diese Kategorie mit nur 1 von 3 Punkten bewertet.  Das Online Angebot ist für die täglichen Herausforderungen von Entwicklern eher ungeeignet, da oft nur Lösungen für Teilprobleme oder Workarounds gesucht werden. Dafür ist das einarbeiten in solche Tutorials zu umständlich, und Vorschläge in üblichen Foren oder Blogs sind aufgrund der kleinen Community oft nicht vorhanden. Das liegt unter anderem auch daran, dass die Dokumentation sehr unstrukturiert ist und es zum Teil deutliche Unterschiede zwischen Version 7 und 8 gibt.

\subsection{Geolocation}

Das ermitteln der Position des Smartphones mittels Windows Phone 8.1 ist durch entsprechende Bibliotheken sehr einfach umzusetzen und erhält deshalb 3 von 3 Punkten in der Bewertungsskala. Sowohl die Abfrage, ob der Benutzer diese Funktionalität zugelassen hat, beziehungsweise zulassen möchte, als auch das asynchrone Abfrage der Position mittels WLAN ist einfach zu implementieren. Die entsprechende Instanz gibt sehr viele Geoinformationen zurück und besitzt auch weitere nützliche Funktionalitäten wie Beispielsweise das errechnen von Distanzen.

\subsection{Notifications}

Für die Implementierung von Benachrichtigungen, sogenannten Push Notifications, wird von Microsoft ausschließlich der Weg über die eigenen Server angeboten, weshalb nur 1 von 3 Punkten vergeben wird. Für die Implementierung von lokalen Benachrichtigungen, werden zwar verschiedene Workarounds vorgeschlagen (Scheduled Tiles, Reminders), jedoch waren diese für die Anforderungen des Projekts eher unbrauchbar.

\subsection{Email}

Zwar ist das verschicken von E-Mails mittels einem ausgewählten Clients ohne Probleme möglich, jedoch können keine Dateien als Anhang hinzugefügt werden. Deshalb wird an der Stelle kein Punkt vergeben. Obwohl die Entwickler-Community das fehlende Feature schon in der Version 7 vermisste, spricht Microsoft weder dieses Problem an, noch wird Stellung bezüglich der Version 10 genommen. Es wird vermutet, dass solch ein Feature möglicherweise aus Sicherheitsgründen nicht erwünscht ist oder nicht als wichtig genug erachtet wird.

\subsection{Persistence}

Das erstellen von lokalen Datenbanken ist mittels LINQ ist sehr einfach umzusetzen, weshalb volle Punktzahl vergeben wird. Mittels der C/# spezifischen Attributes können durch einzelne Klassen und Properties das Design der relationalen Daten sehr einfach erstellt und auch verändert werden. Neben dem Designen ist auch das verwalten der Daten sehr einfach gestaltet. Ein großer Vorteil ist das einbinden von Datenstrukturen in UI Komponenten, was dem Entwickler sehr viel Programmierarbeit erspart. Außerdem können sämtliche Daten mittels sogenannten Lambda-Ausdrücken durchsucht werden. Das macht den Code einerseits sehr lesbar, und die Implementierung einfach. Die Kombination aus C/# spezifischen Funktionalitäten und der SQL Verknüpfung durch LINQ ermöglicht eine optimale Grundlage für die Implementierung von lokalen Datenbanken für Windows Phone. 

\subsection{Design}

\subsection{Testing}

