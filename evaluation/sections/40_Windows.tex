\section{Windows}

Das native Betriebssystem Windows Phone besitzt einen verhältnismäßig kleinen Marktanteil gegenüber iOS und Android. Aufgrund des intensiven Engagements seitens Microsoft und der baldigen Veröffentlichung von Windows 10 (Desktop und Mobile), sollen dennoch Stärken und Schwächen analysiert werden. Für die Evaluierung wurde die neuste Version (Windows Phone 8.1) herangezogen.

\subsection{IDE}

Das Entwickeln von Windows Phone Apps ist ausschließlich mit Visual Studio 2013 (VS) möglich. Die notwendigen Komponenten werden beim sogenannten „Update 2“ mit installiert. Für viele Windows Entwickler ist die Extension „ReSharper“ von der Firma JetBrains essenziel um mit VS produktiv zu arbeiten. Externe Geräte werden sofort von der DIE erkannt und entsprechendes Debugging ist mittels VS sehr elegant gelöst. Im allgemeinen lässt sich mit diesen Entwicklungswerkzeugen sehr gut arbeiten, jedoch sind diese Tools gegenüber anderen Plattformen nicht kostenfrei (ReSharper: 249 Euro, VS Pro: 636 Euro).

\subsection{Language}

Bei Windows ist es üblich sich zwischen Visual Basic oder C\# als Programmiersprache zu entscheiden. Im Rahmen des Projekts wurde ausschließlich C\# verwendet. Solch eine mächtige objekt-orientierte Programmiersprache erleichtert den Entwicklungsaufwand, vorausgesetzt die Entwickler beherrschen diese. Zusätzlich werden Windows Desktop Apps ebenfalls mit C\# entwickelt, weshalb bei Projekten die beide Formate anbieten wollen oder wenn eine Desktop App schon entwickelt wurde, deutlich kürzere Entwicklungszeiten erwartet werden können.

\subsection{Support}

Obwohl Microsoft eine große Auswahl an Tutorials online anbietet und eine Vielzahl an Videos in der sogenannten Microsoft Virtual Academy zur Verfügung stellt wird diese Kategorie mit nur 1 von 3 Punkten bewertet.  Das Online Angebot ist für die täglichen Herausforderungen von Entwicklern eher ungeeignet, da oft nur Lösungen für Teilprobleme oder Workarounds gesucht werden. Dafür ist das einarbeiten in solche Tutorials zu umständlich, und Vorschläge in üblichen Foren oder Blogs sind aufgrund der kleinen Community oft nicht vorhanden. Das liegt unter anderem auch daran, dass die Dokumentation sehr unstrukturiert ist und es zum Teil deutliche Unterschiede zwischen Version 7 und 8 gibt.

\subsection{Geolocation}

\subsection{Notifications}

\subsection{Email}

\subsection{Persistency}

\subsection{Design}

\subsection{Testing}

