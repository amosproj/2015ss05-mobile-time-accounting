\section{Introduction}
Im Rahmen der Veranstaltung AMOS im Sommersemester 2015 war unsere Aufgabe die Entwicklung einer Cross-Platt\-form-Applikation zur Zeiterfassung für Mitarbeiter. Allerdings war die Erstellung einer Evaluation das Hauptziel und die Motivation des Projekts. Hierbei ging es vor allem darum im Laufe des Semesters herauszuarbeiten, ob der Einsatz hybrider Technologien wie HTML5 oder Phonegap eine native Herangehensweise ersetzen kann.

Die Anforderungen an die Zeiterfassungs-App waren also überwiegend daran angelehnt eine interessante und aus\-sage\-kräftige Evaluierung zu ermöglichen. Zu den Kriterien, die von großem Interesse für die Evaluierung waren, zählten die Möglichkeiten der Ortung über GPS oder Wifi, die Erstellung von Notifications, die Nutzung von Standardanwendungen der Betriebssysteme, wie beispielsweise die Email-App sowie nicht-funktionale Kriterien wie die Unterstützung automatisierter Tests, Design-Anpassungen innerhalb der App etc.

Für die Evaluation haben wir für jedes der Kriterien eine Punktzahl vergeben, die sich von 0 bis maximal 3 Punkten erstreckt. In den folgenden fünf Kapiteln, die nach Plattform getrennt sind, soll zu jedem Kriterium eine kurz zusammengefasste Kritik genannt werden, sowie die jeweils erreichte Punktzahl. Nach dem zu jeder Plattform ein Überblick gegeben wurde, soll im siebten Kapitel ein Vergleich erfolgen. Dieses Dokument endet dann mit einem abschließenden Fazit.


