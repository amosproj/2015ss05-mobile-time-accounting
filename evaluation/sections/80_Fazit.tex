\section{Fazit}
Nachdem die vorangegangenen Kapitel die einzelnen Kriterien genauer kommentiert und bewertet haben und anschließend ein Vergleich der Technologien stattfand, soll in diesem Kapitel ein abschließendes Fazit gezogen werden.

Im Rahmen des Projekts und den dabei aufgetretenen Anforderungen bildete HTML5 ganz klar das Schlusslicht. Dies lag aber hauptsächlich daran, dass manche der gewünschten Funktionen, wie beispielsweise Notifications, ohne Backend-Unterstützung nicht realisierbar waren. Unsere Empfehlung für Stand-Alone-Anwendungen ist demnach die native Entwicklung. Ein weiteres Argument, das unsere Empfehlung unterstützt, ist die minimal schlechtere Performanz der hybriden Anwendungen im Bezug auf Datenbankzugriffe. 

Im Bezug auf Client-Server-Applikationen lässt sich jedoch keine eindeutige Aussage treffen, da unser Projekt hierfür nicht geeignet war. Eine Evaluierung der Kommunikationsmechanismen konnte auf Grund des fehlenden Backends nicht durchgeführt werden. Es wäre also durchaus denkbar, dass HTML5 mit Backendunterstützung in manchen Anwendungsfällen punkten kann. Auch der Malus bezüglich Datenbankzugriffen würde bei Client-Server-Applikationen nicht mehr so sehr ins Gewicht fallen, da hier der Übertragungsweg den Engpass darstellt und nicht mehr die gewählte Technologie.
